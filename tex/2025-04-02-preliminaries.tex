This section includes something involved but not detailed in the book.

%\subsection{Infinite}
%\begin{thm}[Monotone Convergence]\label{thm:Monotone-Convergence}
%\end{thm}

\subsection{Abstract Algebra}[From Undergraduate Algebra by Serge Lang]
\begin{defn}[Group]
    A group is a non-empty set $G$ together with a binary operation on $G$, here denoted ``$\cdot$'', that combines any two elements $x$ and $y$ of $G$ to form an element of $G$, such that the following three requirements are satisfied:
    \begin{itemize}
        \item For all $x$, $y$, $z$ in $G$ we have associativity, namely $(x\cdot y)\cdot z = x\cdot (y\cdot z)$.
        \item There exists an element $e$ of $G$ such that $e\cdot x=x\cdot e=x$ for all $x$ in $G$. We call this element the \textbf{unit element} of G. We call it the \textbf{zero} element in the additive case. 
        \item If $x$ is an element of $G$, then there exists an element $y$ of $G$ such that $x\cdot y = y\cdot x = e$. For each $x$, the element $y$ is unique. It is called the \textbf{inverse} of $x$ and is commonly denoted $x^{-1}$.
    \end{itemize}
\end{defn}
One speaks of an \textbf{additive group} whenever the group operation is notated as addition; in this case, the identity is typically denoted $0$, and the inverse of an element $x$ is denoted $-x$. 

Similarly, one speaks of a \textbf{multiplicative group} whenever the group operation is notated as multiplication; in this case, the identity is typically denoted $1$, and the inverse of an element $x$ is denoted $x^{-1}$. In a multiplicative group, the operation symbol is usually omitted entirely, so that the operation is denoted by $xy$ instead of $x\cdot y$.

If this additional condition $x\cdot y=y\cdot x$ holds, then the operation is said to be \textbf{commutative}, and the group is called an \textbf{abelian group}. It is a common convention that for an abelian group either additive or multiplicative notation may be used, but for a nonabelian group only multiplicative notation is used.

For us, typical examples of additive groups $Z$ will be the integers $\mathbf{Z}$, a cyclic
group $\mathbf{Z}_N$, a Euclidean space $\mathbf{R}^n$, or a finite field geometry $F^n_p$. 



\begin{defn}[Cyclic group]
    Let G be a group. We shall say that $G$ is \textbf{cyclic} if there exists an element $a$ of $G$ such that every element $x$ of $G$ can be written in the form $a^n$ for some integer $n$. 
\end{defn}
\begin{exm}
     Consider the additive group of integers $\mathbf{Z}$. Then $\mathbf{Z}$ is cyclic, generated by $1$.
\end{exm}
Let $G$ be a cyclic group 
and let a be a generator. Two cases can occur. 
\begin{itemize}
    \item There is no positive integer $m$ such that $a^m=e$. In this case, we say that G is infinite cyclic, or that $a$ has infinite order. The elements $a^n$ with $n \in \mathbf{Z}$ are all distinct. 
    \item There exists a positive integer $m$ such that $a^m=e$. Then we say that $a$ has finite order, and we call $m$ an exponent for $a$. The smallest positive integer $d$ such that $a^d = e$ is called the period of $a$. If $a^n=e$ then $n = ds$ for some integer $s$. 
\end{itemize}
\begin{lem}
    Let $p$ be a prime number, and $G$ a cyclic group of order $p$ (i.e. the number of elements of $G$ is $p$). Any $a\in G\backslash\{e\}$ is a generator of $G$.
\end{lem}
\begin{proof}
    Let $a\in G$ is a generator of $G$. Then $G=\{e,a,\ldots,a^{p-1}\}$ and $a^p=e$.

    Let $b = a^r\in G$ and $r<p$. $b^p = e$ and $p$ is the period of $b$ since $p$ is a prime number. Let $G'=\{e,b,\ldots,b^{p-1}\}$ is a group of order $p$ (there are not same elements since $p$ is the period of $b$). $G'=\{e,a^r,a^{2r},\ldots,a^{(p-1)r}\}\subseteq G$. So $G'=G$
\end{proof}
\begin{exm}
    Let $p$ be a prime number. $\mathbf{Z}_p = \{0,1,\ldots,p-1\}$ with the binary operation $x\cdot y = x+y\pmod p$. Then $\mathbf{Z}_p$ is a cycle group of order $p$.
\end{exm}

\begin{defn}[Ring]
    A \textbf{ring} $R$ is a set, whose objects can be added and multiplied, satisfying the following conditions:
    \begin{itemize}
        \item Under addition, $R$ is an additive (abelian) group.
        \item For all $x,y,z\in R$ we have $x(y+z)=xy +xz$ and $(y+z)x =yx + zx$.
        \item For all $x,y,z \in R$, we have associativity $(xy)z = x(yz)$.
        \item There exists an element $e\in R$ such that $ex = xe = x$ for all $x\in R$.
    \end{itemize}
\end{defn}

\begin{defn}[Commutative ring]
    A ring $R$ is said to be \textbf{commutative} if $xy = yx$ for all $x, y\in R$.
\end{defn}

\begin{defn}
    A commutative ring without divisors of zero, and such that $1 \neq 0$ is called an integral ring.
\end{defn}
\begin{defn}[Field]
    A commutative ring such that the subset of nonzero elements form a group under multiplication is called a \textbf{field}.
\end{defn}

\begin{exm}
    Some examples of ring or field.
    \begin{itemize}
        \item Integers set $\mathbb{Z}$ is a ring, but not a field.
        \item The rational numbers $\mathbb{Q}$, the real numbers $\mathbb{R}$ are fields.
        \item Finite fields: $\mathbb{Z}_p$ for some prime $p$. For $\mathbb{Z}_n$ for $n$ is not prime, we have some $x\neq 0$ and $y\neq 0$ that $xy=0$. 
    \end{itemize}
\end{exm}

\subsection{General Topology}
\subsubsection{Compactness argument}
[From Lecture Notes~\href{https://math.gmu.edu/~dwalnut/lec03.pdf}{Compactness}, \href{https://virtualmath1.stanford.edu/~conrad/diffgeomPage/handouts/compact.pdf}{compactness criteria}, \href{https://www.math.toronto.edu/ivan/mat327/docs/notes/16-compact.pdf}{Compactness},\href{https://www.math.toronto.edu/ivan/mat327/docs/notes/17-tychonoff.pdf}{Tychonoff's theorem} and wiki.]

A routine way of passing from a finite statement to an infinite one is to use a compactness argument.

\begin{defn}
An open cover of a topological space $X$ is a collection (countable or uncountable) of open sets $\{U_\alpha\}$ such that $X\subseteq \cup_\alpha U_\alpha$. A metric space $X$ is \textbf{compact} if every open cover of $X$ has a finite subcover. Specifically, if $\{U_\alpha\}$ is an open cover of $X$, then there is a finite set $\alpha_1,\ldots,\alpha_N$ such that$X\subseteq \cup_{i=1}^N U_{\alpha_i}$.
\end{defn}
Compactness is a property that seeks to generalize the notion of a closed and bounded subset of Euclidean space.
\begin{thm}[Heine-Borel theorem for $\R^n$]
A subset of $\R$ is compact if and only if it is closed and bounded.
\end{thm}
Tychonoff's theorem states that the product of any collection of compact topological spaces is compact
\begin{thm}[Tychonoff's theorem]\label{thm:Tychonoff's theorem}
Let $X = \prod_{i=1}^\infty X_i$, where each $X_i$ is compact. Then $X$ is compact.
\end{thm}

We present one more ways to think about compactness.
\begin{defn}
We say that a collection $\mathcal{A}$ of subsets of $X$ has the \textbf{finite intersection property} (or \textbf{FIP}) if for every finite collection $\{A_1,\ldots,A_n\}\subseteq \mathcal{A}$, their intersections $\cap_{i=1}^n A_i$ is non-empty.
\end{defn}
\begin{thm}\label{thm:compact-FIP}
Let $X$ be a topological space. Then $X$ is compact if and only if every collection $\mathcal{A}$ of closed subsets of $X$ with the FIP, $\cap\mathcal{A}\neq \emptyset$.    
\end{thm}
\begin{tcolorbox}[pikachu]
To pass from a finite statement to an infinite one, we first prove $X$ is compact (By Theorem~\ref{thm:Tychonoff's theorem}) then prove collection $\mathcal{A}$ has the FIP, then we have $\cap\mathcal{A}\neq\emptyset$. See the proof of Theorem~\ref{thm:colorings-of-the-real-line} as an example.
\end{tcolorbox}

%\subsection{Probability}

\subsection{Distribution of Primes}
In this subsection, we shall adopt the convention that whenever a summation is over the index $p$, then $p$ is understood to be prime.


The distribution of the primes $\P=\{2,3,5,\ldots\}$ is a very well-studied subject in analytic number theory, with one of the fundamental results being the prime number theorem.

\begin{thm}[Prime number theorem]
\[
    |P\cap[1,n]|=(1+o(1))\frac{n}{\log n}.
\]
An equivalent formulation is that if $p_k$ denotes the $k$th prime, then $p_k = (1 +o(1))k \log k$.
\end{thm}

\begin{prop}[Elementary prime number estimates]\label{prop:Elementary-prime-number-estimates}
Let $n\geq 1$ be an interger. Then we have the estimates
\begin{align}
\sum_{p\in n}\log p &= O(n)\\
\sum_{p\in n}\frac{\log p}{p} &= \log n+O(1)\\
\sum_{p\in n}\frac{1}{p} &= \log\log n+O(1).
\end{align}
\end{prop}

\begin{thm}\label{thm:the-distribution-of-primes-in-intervals}
For all sufficiently large $n$, we have $|P\cap [n-x,n)|= \Theta( \frac{x}{\log n})$ for all $n^{2/3}<x<n$.
\end{thm}

\begin{prop}\label{prop:Elementary-prime-number-estimates-in-intervals}
Let $n$ be a large integer. Then we have the estimates 
\begin{align}
\sum_{p\in P\cap[1,n-n^{2/3})}\frac{1}{n-p}&=\Theta(1)\\
\sum_{p\in P\cap[1,n-n^{2/3})}\frac{\log(n-p)}{n-p}&=\Theta(\log n).
\end{align}

\subsection{Probability theory}
\begin{lem}[Law of total expectation]\label{lem:Law-of-total-expectation}
If $X$ is a random variable whose expected value $\E(X)$ is defined, and $Y$ is any random variable on the same probability space, then we have
\[
\E(\E(X|Y))=\E(X).
\]
\end{lem}
\begin{tcolorbox}[pikachu]
The conditional expected value $\E(X | Y)$, with $Y$ a random variable, is not a simple number; it is a random variable whose value depends on the value of $Y$. That is, the conditional expected value of $X$ given the event $Y = y$ is a number and it is a function of $y$. If we write $g(y)$ for the value of $\E(X | Y = y)$ then the random variable $\E(X | Y)$ is $g(Y)$.    
\end{tcolorbox}
\end{prop}